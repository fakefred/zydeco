\usepackage{iftex}
% \ifPDFTeX
%   \usepackage{libertinust1math}
%   \usepackage{libertine}
% \else
%   \usepackage{libertinus}
% \fi
% \usepackage[scaled=0.8]{FiraMono}
\usepackage{indentfirst}
% \usepackage{listings}
\usepackage[newfloat]{minted}
\usepackage{etoolbox}
\AtBeginEnvironment{minted}{\singlespacing%
    \fontsize{10}{10}\selectfont}
\setminted{baselinestretch=0.7}
\setminted{fontsize=\footnotesize}

\usepackage{caption}

% \newenvironment{code}{\captionsetup{type=listing}}{}
% \SetupFloatingEnvironment{listing}{name=Source Code}

\usepackage{xcolor}

\usepackage{amsmath}
% \usepackage{amssymb}
\usepackage{amsfonts}
\usepackage{amsthm}
\usepackage{thmtools}
\usepackage{mathtools}
\usepackage{mathpartir}
% \usepackage{stmaryrd}
\usepackage{bm}

\usepackage{multicol}

\usepackage{tikz}

\usepackage{booktabs}
\usepackage{enumitem}
\setlist{itemsep=1pt}

% \usepackage{biblatex}

% \usepackage[colorlinks=true, linkcolor=purple]{hyperref}
\usepackage[capitalise, noabbrev]{cleveref}

\usepackage{subfiles}

\numberwithin{figure}{section}
\numberwithin{table}{section}

\makeatletter
\renewcommand\tableofcontents{\@starttoc{toc}}

\declaretheorem[name=Theorem, parent=section]{theorem}
\declaretheorem[name=Definition, parent=section, style=definition]{definition}

% % %
% Judgments
%
\newcommand{\judgment}[3]{\inferrule{#1}{#2}~~{\textsc{\small [#3]}}}
\newcommand{\judgbox}[1]{\noindent \fbox{$#1$}}

% Contexts
\newcommand{\ctx}{\ensuremath{\Gamma}}
\newcommand{\extendCtx}[3]{\ensuremath{#1 ; ~\assignType{#2}{#3}}}
\DeclareMathOperator{\dom}{dom}
\newcommand{\inCtx}[3]{\ensuremath{#2, #3 \in #1}}
\newcommand{\notInCtx}[2]{\ensuremath{#2 \not\in \dom(#1)}}
\newcommand{\withCtx}[2]{\ensuremath{#1 \vdash #2}}

% Typing
\newcommand{\assignType}[2]{\ensuremath{#1 : #2}}
\newcommand{\ctxAssignType}[3]{\ensuremath{\withCtx{#1}{\assignType{#2}{#3}}}}
\newcommand{\synType}[2]{\ensuremath{#1 \Rightarrow #2}}
\newcommand{\ctxSynType}[3]{\ensuremath{\withCtx{#1}{\synType{#2}{#3}}}}
\newcommand{\anaType}[2]{\ensuremath{#1 \Leftarrow #2}}
\newcommand{\ctxAnaType}[3]{\ensuremath{\withCtx{#1}{\anaType{#2}{#3}}}}

%
% Types
%

% Value types
\newcommand{\TAMName}{\textsf{Value Type}}
\newcommand{\TAMV}{\ensuremath{A}}
\newcommand{\TAInt}{\ensuremath{\textsf{Int}}}
\newcommand{\TABool}{\ensuremath{\textsf{Bool}}}
\newcommand{\TAString}{\ensuremath{\textsf{String}}}
\newcommand{\TAThunk}[1]{\ensuremath{U ~#1}}

% Computation types
\newcommand{\TBMName}{\textsf{Computation Type}}
\newcommand{\TBMV}{\ensuremath{B}}
\newcommand{\TBReturn}[1]{\ensuremath{F ~#1}}
\newcommand{\TBArrow}[2]{\ensuremath{#1 \to #2}}

%
% Values
%
\newcommand{\AMName}{\textsf{Value}}
\newcommand{\AMV}{\ensuremath{a}}

% Literals
\newcommand{\ALit}[1]{\ensuremath{\underline{#1}}}
\newcommand{\AInt}{\ensuremath{\ALit{n}}}
\newcommand{\AString}{\ensuremath{\ALit{s}}}
\newcommand{\ATrue}{\ensuremath{\textsf{tt}}}
\newcommand{\AFalse}{\ensuremath{\textsf{ff}}}

% Thunks
\newcommand{\AThunk}[1]{\ensuremath{\{ #1 \}}}

%
% Computations
%
\newcommand{\BMName}{\textsf{Computation}}
\newcommand{\BMV}{\ensuremath{b}}

% Binding
\newcommand{\BLet}[3]{\ensuremath{\textsf{let}~ #1 = #2 ~\textsf{in}~ #3}}
\newcommand{\BLetAnn}[4]{\ensuremath{\textsf{let}~ #1 : #2 = #3 ~\textsf{in}~ #4}}
\newcommand{\BLetRec}[4]{\ensuremath{\textsf{let rec}~ #1 : #2 = \AThunk{#3} ~\textsf{in}~ #4}}
\newcommand{\BDo}[3]{\ensuremath{\textsf{do}~ #1 \gets #2 ~\textsf{in}~ #3}}
\newcommand{\BDoAnn}[4]{\ensuremath{\textsf{do}~ #1 : #2 \gets #3 ~\textsf{in}~ #4}}

% Lambda
\newcommand{\BLam}[2]{\ensuremath{\lambda #1. ~#2}}
\newcommand{\BLamAnn}[3]{\ensuremath{\lambda #1 : #2. ~#3}}
\newcommand{\BAp}[2]{\ensuremath{#1 ~#2}}

% Force
\newcommand{\BForce}[1]{\ensuremath{! #1}}

% Return
\newcommand{\BReturn}[1]{\ensuremath{\textsf{return}~ #1}}

% Conditional
\newcommand{\BIf}[3]{\ensuremath{\textsf{if}~ #1 ~\textsf{then}~ #2 ~\textsf{else}~ #3}}

% Data & Codata
\newcommand{\DeclName}{\textsf{Declarations}}
\newcommand{\VPName}{\textsf{Var-Param}}
\newcommand{\VPV}{\ensuremath{v^*}}
\newcommand{\MatcherName}{\textsf{Matcher}}
\newcommand{\ComatcherName}{\textsf{Comatcher}}
\newcommand{\MatcherV}{\ensuremath{\varphi_m}}
\newcommand{\ComatcherV}{\ensuremath{\varphi_c}}
